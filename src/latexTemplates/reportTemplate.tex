\documentclass[12pt]{article}
\usepackage{graphicx}
\usepackage{amsmath} 
\usepackage[utf8]{inputenc} 
\usepackage{booktabs,multirow,multicol}
\usepackage{subcaption}
\usepackage{datetime}
\usepackage{float} 
\usepackage{hyperref}
\usepackage{verbatim}
\usepackage{siunitx} 
\usepackage{enumerate} 
\usepackage{gensymb}
\setlength\parindent{0pt}
 
\newdateformat{Date}{
Bogotá, D. C.\hspace*{5pt} \monthname[\THEMONTH] \THEDAY, \THEYEAR}
\newcommand{\Client}{$CLIENTE }
\newcommand{\Sample}{$SAMPLE }
\newcommand{\SampleID}{$ID}

\begin{document}

\Date\today \newline
\vspace{30 pt}

Señores:\newline
\vspace{10pt}
\Client \\
Bogotá, D. C.  - COLOMBIA

\vspace{50 pt }

\textbf{Ref: INFORME TECNICO ANALITICO CRUDE ASSAY TIPO III \Sample}\newline

\vspace{20pt}

De antemano reciban un cordial saludo y los agradecimientos por haber
depositado la confianza en nuestro laboratorio, esperam os seguir siendo su
proveedor para análisis geoquímicos de confianza.\\

Estamos haciendo entrega del informe de resultados para los análisis de \Sample practicados a la muestra de crudo correspondiente a:
\vspace{20pt}
\begin{table}[H]
\centering
   \begin{tabular}{|c|c|}
      \hline
   ID Muestra &  Nombre \\
    \hline \hline
    \SampleID & \Sample  \\
    \hline
   \end{tabular}
\end{table}
\vspace{20pt}
Cualquier tipo de inquietud que se presente, estaremos atentos a responderle.
\vspace{30pt }

\textbf{HYDROCARBON Q.A}\\
Empresa encargada.
\newpage
\begin{center}
\begin{Huge}
\textbf{CRUDE ASSAY TIPO III\\ \Sample}\\
\end{Huge}
\begin{Large}
\vspace{70pt}
\textbf{INFORME $\mathbf{N^{\circ}}$ }\\
\textbf{\SampleID}\\
\vspace{100 pt}
\textbf{PREPARADO PARA:}\\
\Client\\
\vspace{100 pt}
\textbf{PRESENTADO POR:}\\
HYDROCARBON Q.A\\
\vspace{45pt}
\textbf{\Date\today}
\end{Large}
\end{center}
\newpage
\section*{RESUMEN EJECUTIVO}
\vspace{30 pt }

A continuación se presenta el informe técnico con los resultados de la
caracterización Crude Assay tipo III, efectuada al \Sample, para la compañía \Client. El proceso de destilación se
desarrolla siguiendo los protocolos analíticos referenciados en las normas ASTM
D2892 Y ASTM D5236.\\

El análisis consiste en la caracterización fisicoquímica del crudo, y posterior
destilación de 13 cortes, cuyos rangos de ebullición para el fraccionamiento y
correspondientes rutinas analíticas, fueron estipulados por \Client bajo
acuerdo. 
\newpage
\section*{INTRODUCCIÓN}
El conocimiento de cada uno de los fluidos que participan en la cadena de valor de
los hidrocarburos, es de capital importancia para el desarrollo, rendimiento,
compatibilidad y comercialización de las fracciones a obtener en un proceso de
refinación.\\

El presente informe contiene los resultados para la caracterización analítica
“Crude Assay III”, donde se desarrolla la destilación física de una muestra de
crudo entero y sus diversos cortes en el intervalo de ebullición, de acuerdo a los
protocolos establecidos en las normas ASTM D-2892 y D-5236.\\

El proyecto inicia con la recepción de la muestra de crudo deshidratado "\Sample" en las instalaciones de HYDROCARBON Q.A, se da ingreso a la fracción de crudo entero y posteriormente se practica la siguiente rutina analítica:
\begin{enumerate}
\item Caracterización petrobásica.
\item  Destilación atmosférica hasta $199^{\circ}C $, obtención de los cortes de
hidrocarburos gaseosos y naftas, siguiendo los lineamientos de la norma
ASTM D2892.
\item Destilación con vacío hasta $371^{\circ}C $, obtención de los cortes medios y crudo
reducido; siguiendo los lineamientos de la norma ASTM D2892.
\item Destilación con alto vacío del crudo reducido hasta $565^{\circ}C $, obtención de las
fracciones de gasóleo, y fondo; siguiendo los lineamientos de la norma
ASTM D5236.
\item Caracterización fisicoquímica de cada una las fracciones obtenidas
Los programas de procedimientos, protocolos analíticos, resultados e
interpretación de los mismos, son presentados a continuación.
\end{enumerate}

\newpage
\begin{center}
\section*{OBJETIVOS}
\end{center}
\section*{OBJETIVO GENERAL}
\vspace{20 pt}
Realizar caracterización “Crude Assay III” al crudo entero deshidratado
“\Sample”, siguiendo los protocolos analíticos de las normas
ASTM D 2892 y D5236.
\section*{OBJETIVOS ESPECIFICOS}

Caracterizar las principales propiedades petrobásicas a la fracción de crudo
entero.\\

Realizar destilación atmosférica hasta $199^{\circ}C $; siguiendo los lineamientos de la
norma ASTM D2892.\\

Realizar destilación con vacío (100 y 10 mmhg) hasta $371^{\circ}C $; siguiendo los
lineamientos de la norma ASTM D2892.\\

Realizar destilación con alto vacío (0.5 mmhg) del crudo reducido hasta  $565^{\circ}C $;
siguiendo los lineamientos de la norma ASTM D5236.\\

Realizar caracterización fisicoquímica establecida para cada uno de los cortes
obtenidos.\\

Integrar la información obtenida de la destilación y realizar la interpretación de
resultados asociada al rendimiento, calidad y composición molecular de cada
corte.
\newpage

\begin{center}
\section*{RESULTADOS }
\end{center}


\vspace{20 pt}
A continuación se presentan los resultados asociados a los rendimientos y
caracterización fisicoquímica para cada una de las fracciones del crudo
\Sample. \\

\section*{Balance de materia elementos no metálicos}
\vspace{20 pt}
Después de realizar la ecuación correspondiente al balance de materia de los datos ingresados en el sistema para los elementos metálicos, se obtuvieron los siguientes resultados: \\ 
\vspace{20 pt}
\begin{table}[H]
  \centering
    \begin{tabular}{|r|r|}
    \hline
    Elemento & Balance \\
    \hline\hline
    Aluminio &  $results[0] \\
    Níquel  &  $results[1] \\
    Vanadio &  $results[2]\\
    Hierro & $results[3] \\
    Cobre  &  $results[4] \\
    Magnesio &  $results[5] \\
    Calcio &  $results[6] \\
    Sodio &  $results[7] \\
    \hline
    \end{tabular}%
\end{table}%
\vspace{30 pt}
\section*{Balance de materia elementos metálicos}
\vspace{20 pt}
Después de realizar la ecuación correspondiente al balance de materia de los datos ingresados en el sistema para los elementos no metálicos, se obtuvieron los siguientes resultados: \\
\vspace{20 pt}
\begin{large}
\begin{table}[H]
  \centering
    \begin{tabular}{|r|r|}
    \hline
    Elemento & Balance \\
    \hline\hline
    Azufre &  $results[8] \\
    Cenizas  &  $results[9] \\
    Asfaltenos & $results[10] \\
    \hline
    \end{tabular}%
\end{table}%
\end{large}

\newpage
\section*{Test de viscosidad}

Después de realizar un análisis  a los valores de las viscosidades a temperaturas de $40^{\circ}C $,$50^{\circ}C $ $80^{\circ}C $, $100^{\circ}C $ y $120^{\circ}C $ para cada una de las fracciones que componen la muestra \textbf{\Sample}; se determinó que los valores de las mismas \textbf{$VISCOSIDAD se encuentran dentro de los rangos aceptables  establecidos.}\\

Lo anterior se afirma al  $VISCOSIDAD evidenciar un aumento en los valores de las viscosidades conforme se analiza cada fracción dado el siguiente orden:

\begin{itemize}
\item Nafta 1 
\item Nafta 2
\item Nafta 3
\item Nafta 4
\item Medio 1
\item Medio 2
\item Medio 3
\item Gasóleo 1
\item Gasóleo 2
\item Gasóleo 3
\item Fondo de vacío
\end{itemize}
\newpage

\section*{Cálculo del nitrógeno total }

Para realizar el cálculo del nitrógeno total para cada fracción, se consultó la base de datos de históricos, de la cual se seleccionó la muestra cuya densidad API fuese la más cercana a la de la fracción analizada. Posteriormente, utilizando un modelo de regresión lineal simple, se determinaron los parámetros $\beta_{0}$ y $\beta_{1}$ de la función para el ajuste de la curva, la cual fue implementada para estimar  el valor del nitrógeno correspondiente a cada fracción.\\

A continuación se presentan los resultados obtenidos de implementar el procedimiento anteriormente descrito.
\vspace{40 pt}
\begin{large}
\begin{table}[H]
  \centering
    \begin{tabular}{|r|r|}
    \hline
     Fracción & Nitrógeno Total \\
    \hline\hline
     Nafta 1  & $results[11]  \\
     Nafta 2 & $results[12]  \\
     Nafta 3 & $results[13]  \\
     Nafta 4 & $results[14]  \\
     Medio 1 & $results[15]  \\
     Medio 2 & $results[16] \\
     Medio 3 & $results[17]\\
     Gasóleo 1 & $results[18] \\
     Gasóleo 2 & $results[19] \\
     Gasóleo 3 & $results[20]  \\
     Fondo de vacío & $results[21]  \\
    \hline
    \end{tabular}%
\end{table}%
\end{large}

\end{document}